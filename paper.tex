\documentclass[letterpaper, 12pt]{article}
%%\documentclass[letterpaper, 11pt, twocolumn]{article}
\usepackage{url}
\usepackage{ctable}
\usepackage{overcite}
\usepackage{amsmath,amssymb}      % for \AA etc
\usepackage[onehalfspacing]{setspace}
\usepackage{graphicx}
\usepackage{fullpage, times}
\usepackage{algorithmic}
\usepackage{algorithm}
\usepackage{longtable}

\DeclareGraphicsExtensions{.pdf,.png}
\onehalfspacing

\bibliographystyle{jcics}

\usepackage{anysize}
%\marginsize{1in}{2in}{1in}{1in}
%\marginparwidth 1.75in
%\newcommand{\mnote}[1]{\marginpar{\raggedright{\footnotesize\textbf{#1}}}}

\def\imagetop#1{\vtop{\null\hbox{#1}}}

%%%
%%% Don't change the size of the figures as they are defined for
%%% a single column environment, which is what we will be submitting
%%% as
%%%

\begin{document}
\title{A Survey of Quantitative Descriptions of Molecular Structure}
\author{Rajarshi Guha${}^{\dagger}$ and Egon Willighagen${}^{\ddagger}$\\
${}^{\dagger}$NIH Center for Advancing Translational Science\\9800 Medical
Center Drive  Rockville, MD 20850 \\U.S.A.\\
${}^{\ddagger}$ Department of Bioinformatics - BiGCaT\\Maastricht University\\P.O. Box 616\\NL-6200 MD Maastricht\\The Netherlands
}
\date{}

\maketitle
\begin{abstract}
Bar
\end{abstract}

\section{Introduction}

Computational methods play an important role in many chemical
disciplines ranging from drug discovery to materials science. While
there are a variety of techniques that differ in terms of
computational complexity, time requirements and so on. However the
common requirement underlying all these methods is a formal
description of a the molecular structure. Of course, there are many
ways to ``describe'' a molecule. A common approach is to describe the
connectivity, taking into account the types of atoms and bonds. In
other words, explicit representations of chemical structure such as
SMILES, SD files and so on. While these descriptions are vital to
modern chemical information systems, they do not necessarily allow
computational techniques to be directly applied to them.

Methods that aim to predict chemical and biological properties
generally require a numerical description of chemical structures. Such
numerical forms range from a set of 3D coordinates (which coupled with
appropiate atom types, is sufficient for methods such QM approaches
and docking) to more abstract numerical descriptions derived from 2D
or 3D representations which can be useful is statistical approaches.

It is now possible to evaluate thousands of numerical descriptors of
chemical structure. As will be discussed later, many of these
descriptors are closely related - one can be substituted for
another. The choice of descriptors is a well known problem and given a
large collection of them, approaches to identify a suitable subset  
have been discussed extensively in the literature [REFS]. 

In addition to there being many possible descriptors defined in the
literature, there are also multiple implementations of a give
descriptor. These implementations are available in the form of
libraries (which require one to write a program to use them) or
complete applications (GUI or command line). As a result, not only
must one choose one or more descriptors that are relevant to the
problem at hand, but in addition, one must be concerned about how they
are calculated and whether such a calculation can be reproduced across
different implementations of those descriptors. It is easy to
understand \emph{why} two implementations of the same descriptor can
lead to different results - the primary reason being the chemistry
model that underlies the framework or toolkit used to implement the
descriptor. For example, a descriptor that calculates the number of
aromatic atoms may be implemented using two toolkits with differing
aromaticity models, and hence it is possible that the values generated
by the two implementations will differ. Other sources of differences
include parameters that may be involved in the descriptor calculation
and reference data values (such as atomic radii, electronegativity
values) that are employed during descriptor calculation. While most
implementations will employ the same data sources for standard
concepts (e.g., atomic weights), slight differences in these types of
input data can lead to differences in the final descriptor value. As a
result, in most cases, two implementations of a descriptor do not
usually give the exact same value, though they are usually quite
similar. Explicitly explaining the differences may or may not be
possible (it is usually more difficult in the cases of commercial
implementations for which source code is not available).

\section{A Categorization of Descriptors}
\label{sec:categ-descr}

We have noted that a molecular structure can be characterized using a
numebr of numerical descriptors. In this section we describe a
categorization of descriptors. Admittedly, the grouping is somewhat
arbitrary but serves as a useful guide. In addition, the
categorization also considers the nature of the chemical structure
being considered - some descriptors are only useful when applied to
small molecules, whereas other descriptors are defined specifically
for polymers or protein structures.

Broadly, one can classify descriptors based on the nature of
structural information they require. Constitutional descriptors only
require atom and bond labels and usually represent counts of different
types of atoms or bonds. While very simplistic they can play a useful
role in a variety of applications ranging from summaries of
physicochemical properties to predictive modeling
\cite{Bender:2005aa}.  

Topological descriptors take into account connectivity along with atom
and bond labels. These descriptors essentially consider the molecule
as a labeled graph and characterize it using graph invariants. In
addition to graph invariants, topological descriptors based on
information theory such as entropy \cite{Dehmer:2009uq}. An advantage
of these descriptors is that do not require any intensive
pre-processing steps such as 3D coordinate generation and
conformational analysis. In addition, many graph invariants can be
rapidly computed for graphs corresponding to small molecule structures
(even though they may be computationally intractable on much larger
graphs). While this class of descriptors have been used extensively in
predictive modeling applications
\cite{Garcia-Domenech:2008aa,Randic:2001ac,Randic:2001ad,Besalu:2001aa,jcics:1995:35:272,Kier:1986ae},
a key drawback is their lack of interpretability. Given the abstract
nature of many graph invariants and information theoretic descriptors,
their use in predictive models makes it difficult to explain the model
in simple physicochemical terms, though various attempts have been  
made at interpretations \cite{Todeschini:1975dq,Stanton:2003aa}.

Geometric descriptors require a 3D conformation as input and therefore
are not as fast as topological descriptors which do not require the 3D
coordinate generation step. Nearly all geometric descriptors work on a
single conformation and for cases where multiple conformation must be
considered, an averaging procedure is usually employed. Examples of
these descriptors include the gravitational index
\cite{Katritzky:1996ly} and moment of inertia descriptors. However,
geometric descriptors encompass many aspects of a molecular
structure. For example, a number of geometric descriptors characterize
the molecular surface. The simplest such example is the surface area
descriptor (and the corresponding volume descriptor). A simple
extension of such a descriptor is to characterize distributions of a
physicochemical property over the molecular surface such as the
Charged Partial Surface Area descriptors \cite{Stanton:1990aa} that
characterize the partial charge distribution over the molecular
surfce. Note that surface descriptor calculations based on analytical
\cite{Connolly:1983aa} or tesselation algorithms
\cite{Eisenhaber:1995qf} can be slow and that parametrized methods
such as the Topological Polar Surface Area (TPSA) \cite{Ertl:2000aa}
and VSA descriptors \cite{Labute:2008aa} that are based on
precalculated surface area values derived from a list of functional
groups. A comparison of surface area values generated by the the
tesselatin method and the VSA method (as implemented in the CDK) for
XXX molecules indicates a high degree of correlation.

Molecular shape is a key factor in ligand-receptor interactions and is
an important approach to virtual screening. A variety of descriptors
have been devised to characterize molecular shape. The use of
overlapping atom centered Gaussians to represent molecular shape
\cite{Grant:1995aa} represents a significant computational advantage
over the traditional analytical approach of intersecting hard spheres,
supporting extremely rapid shape comparisons \cite{Grant:1999vn}. An
alternative approach described by Ritchie et al \cite{Ritchie:1999kx}
employed a spherical harmonic based representation of surfaces, that
in contrast to the Gaussian approach allows one to generate
characterizations of the surface and hence the shape at varying
resolutions. Another approach to multi-resolution shape
representations is the use of $\alpha$ shapes
\cite{Edelsbrunner:1994aa}, which represent a parametrization of the
convex hull derived from the molecular coordinates. This approach has
been employed to characterize shape and volume \cite{Wilson:2009ys}
though it is more computationally intensive than the methods mentioned
previously. In addition to functional representations of molecular
shape, a number of simpler formulations have been defined. For
example, the ``shape signature'' approach described by Zauhar et al
\cite{Zauhar:2003fk} is a novel application of a ray tracing method
that generates a description of the molecular shape. Ballester et al
\cite{Ballester:2007aa} described the Ultrafast Shape Recognition
(USR) approach that characterized molecular shape in terms of the
distribution of distances between the atoms and a set of pre-defined
points. This procedure characterizes the shape in the form of a 12-D
real valued fingerprint. As a result, the calculation procedure is
rapid and once calculated shape similarities can be evaluated rapidly
via a simple distance calculation. It should also be noted that shape
indices based on the molecular topology have also been described
\cite{Kier:1985aa}, which are very rapid to calculate (somewhat
analogous to the VSA descriptors) but do not directly represent the
molecular shape and instead

Finally, quantum mechanical calculations can also generated
descriptors that can be applied to problems such as QSAR models
(though of course the time complexity of QM derived descriptors do not
make them amenable for large scale virtual screening
applications). These descriptors can be evaluated at various levels of
theory. For example, partial charges, HOMO and LUMO energies are
commonly evaluated using semi-empirical methods, while Eroglu and
T\"{u}rkmen \cite{Eroglu:2007uq} described DFT-derived descriptors
such as electronegativity, hardness and ionization potential (though
Puzyn et al question the need for such a high level of theory
\cite{Puzyn:2008fk}).

The descriptor categories listed here are not necessarily distinct,
many hybrid descriptors have been described that combine one or more
categories. For example, the charge indices described by Galvez et al
\cite{Galvez:1994aa} combine partial charge information and
topological connectivity. Similarly the CPSA descriptors combine
surface area and partial charge information. When considering hybrid
descriptors, it is useful to note the autocorrelation descriptors that
characterize the distribution of molecular properties across a
molecule in terms of an autocorrelation function. These descriptors
may be based on topological or geometric distances and can be combined
with arbitrary molecular properties \cite{Moreau:1980aa,Broto:1984aa}.

\textbf{Fingerprints} Traditionally, fingerprint descriptors are
represented in the form of bit strings. Binary fingerprints can be
divided into hashed fingerprints, where substructures (such as paths
of length $n$) are converted to a string representation and then
hashed into a randomly selected bit positions, or keyed fingerprints,
where each bit position corresponds to a unique substructural
feature. The nature of the features can range from simple functional
groups (hydroxyl, carbonyl) and topological substructures (paths,
chains, cycles) to atom environments \cite{Bremser1978,Bender:2004aa}
and pharmacophoric elements \cite{Renner:2006aa}. Depending on the
definition, fingerprints can work with topology only or may require 3D
conformations. The latter class of fingerprints are usually related to
3D pharmacophores and most fingerprint definitions only require
connectivity information. 

Fingerprints are a compact representation of a chemical structure and
are used in a variety of ways ranging from database searches looking
for similar structures or substructures to virtual screenings studies
\cite{Sun:2006kx}. In many search applications, the binary nature of
fingerprints allow for efficient algorithms to compare molecules and
evaluate similarity and a number of heuristics have been developed
that allow for rapid similarity searches
\cite{Baldi:2008aa,Swamidass:2007ab}. In virtual screening scenarios,
fingerprints can be used as features. In many cases, the entire
fingerprint is employed as a set of independent variables. Since this
can lead to poor predictive models, some sort of feature selection is
usually employed and there have been studies that address the problem
of identifying subsets of bits from a fingerprint that lead to optimal
performance in predictive models (albeit, usually in a target-class
specific manner) \cite{Xue:2004fk}.

We next consider descriptors that are designed for specific classes of
molecular systems.

\textbf{Descriptors for polymeric systems} Polymeric systems represent
a challenge for descriptor development. There have been reports that
employed tradtional small-molecule descriptors
\cite{Mattioni:2002dq,Katritzky:1998cr} that consider the polymer in
terms of the individual monomers. A number of approaches have been
described that includes both deterministic, geometry derived
descriptors \cite{Edvinsson:2003bh} as well as stochastic methods
\cite{Gonzales-Diaz:2003ly}. For the case of polypeptides the
descriptors generally tend to be based on the individual amino
acids. Examples include the VHSE family of descriptors described by
Mei et al \cite{Mei:2005kx}, a set of side chain descriptors from
Collantes and Dunn \cite{Collantes:1995vn}, field based descriptors
\cite{Norinder:1991ys} and the AAindex database
\cite{Kawashima:1999zr} which allows rapid lookup of amino acid
properties.

\textbf{Descriptors for inorganic compounds} Inorganic materials are
sufficiently different from small molecules that traditional
descriptors are not very useful. Recently, a number of descriptors for
inorganic materials are very similar to constitutional descriptors -
essentially counts of various crystal parameters such as acccessible
surface areas and volumes
\cite{Willems:2012zr,Haranczyk:2010ys,Mackay:1984ve}. Other
descriptors characterize the periodic voids of porous materials via
tesselation methods \cite{Anurova:2009ly,Carr:2009kx}

\textbf{Bioactivity as a descriptor} While the descriptors mentioned
so far are derived from some form of chemical structure, an
alternative approach is to employ observed biological activities of a
molecule as descriptors of that molecule.  This approach was taken by
Sedykh et al \cite{Sedykh:2011fk} who hypothesized that dose response
data points from high throughput dose-reponse assays could be employed
as biological descriptors. They observed that when such descriptors
were combined with traditional chemical structure descriptors, the
predictive performance of models developed using the combined set was
better than those developed using conventional approaches. A similar
approach can be taken using the PASS methodology
\cite{Poroikov:2007aa}, though this is an indirect method - one must
predict the PASS profile from the chemical structure, which can then
be used in subsequent predictive models.

\section{Descriptor Implementations}
\label{sec:descr-impl}

A variety of descriptor implementations are available, with most
cheminformatics toolkits providing implementations of a common set of
descriptors and fingerprints. Table \ref{tab:impl} summarizes the
descriptor classes implemented by several toolkits and applications. 

While library calls for descriptor evaluation is the most flexible
approach, it is not useful for non-programmers as well as users who
simply need a descriptor matrix for modeling or searching. As a
result, GUI or command line wrappers around descriptor APIs enable
easier usage. The CDKDescUI application
(\url{http://rguha.net/code/java/cdkdesc.html}) is a Java Swing
application that exposes the CDK descriptor and fingerprint API's. It
allows the users to specify a SMILES or SD file and evaluate a
selected set of descriptors or fingerprints and export them to a text
file. The application also has a command line mode that can be run in
server environments. A similar application is PaDEL \cite{Yap:2011fk}
that includes the CDK descriptors and also implements a number of
additional descriptors such as the extended topochemical atom
descriptors \cite{Roy:2004uq}, McGowan Volume \cite{Zhao:2003kx}, and
the Klekota \& Roth substructure counts \cite{Klekota:2008vn}.

In addition to libraries and their wrappers there are various GUI
applications that allow easy descriptor calculations. Examples include
Bioclipse \cite{Spjuth:2007aa}, MOE (Chemical Computing Group) and
Maestro (Schr\"{o}dinger, Inc.). Workflow tools are also a useful
class of applications for descriptor generation and allow one to
easily generate descriptors from multiple sources.

\subsection{Comparing implementations}
\label{sec:comp-impl}


Given the multiple implementations of many descriptors, it is natural
to compare their performance. As noted previously, it is desirable,
but unlikely that implementations of the same descriptor using
different cheminformatcs toolkits will be identical. Gupta et al
employed SMARTS based descriptors from the CDK and MOE to develop
decision tree models to predict human liver microsomal metabolic
stability\cite{Gupta:2010uq}. Their results indicated very similar
performance between the two implementations. 

One can also directly compare descriptor values generated using
different implementations. 

A number of physicochemical descriptors commonly employed in
predictive modeling are themselves models of an experimental
property. For example, log P can be experimentally measured and a
number of algorithms have been developed to predict log P
[REFs]. Given the utility of log P in drug discovery, a number of
implementations are available. Given that this descriptor is a
surrogate for an actual experimental property, it is reasonable to
compare calculated log P values from different implementations to the
experimentall observed values, rather than between themselves. Figure
\ref{fig:logp} compares computed log P values from the CDK, ChemAxon
and ACD Labs for a set of 10,000 molecules taken from the logPstar
dataset [REF].

\subsection{Descriptor Naming \& Versioning}
\label{sec:descr-vers}

As noted above, different implementations of the same descriptor can
generate slightly different values depending on the underlying
chemistry model, differences in reference values and so on. Even
within the same software implementation, different versions of the
same descriptor can generate different values. Thus to maintain
reproducibility in sudies that employ molecular descriptors, some form
of versioning is crucial. The simplest form of version is to manually
note the version of the implementation used for the calculation and
report it along with the results. Though useful, it is not amenable to
automation. In some cases, toolkits will provide an API call to
retrieve the version information, which can then be associated with a
descriptor calculation. 

A topic closely related to versioning is descriptor naming. The
preceding sections have discussed a categorization of descriptors and
have used a number of commonly used names. When performing
calculations in an automated manner, or employing remote services for
descriptor calculations \cite{Dong:2007aa}, it is useful to be able to
refer to descriptors via a standardized naming scheme. Currently the
QSAR modeling community lacks such a scheme, though the Chemistry
Developmen Kit (CDK), a Java library for cheminformatics
\cite{Steinbeck:2006aa} does implement such a scheme. The ``descriptor
specification'' approach defines a set of classes that include
information of the implementation title, identifier and vendor. The
identifier may or may not incude a version number. The actual details
of the a descriptor are listed in a dictionary, along with an optional
namespace. The dictionary entry is referenced by the descriptor
specification and this design thus allows one to seamlessly work with
multiple implementations of the same descriptor from different vendors
and keep track of which version is employed throughout a calculation
or workflow. In addition, the CDK provides a hierarchical annotation
of descriptors, grouping them into different classes. Since this
annotation is based on RDF \cite{Taylor:2006ab}, it allows one to go
beyond simple groupings and perform reasoning on descriptors. A simple
example is to identify functionally equivalent descriptors (such as
simple transformations of a base descriptor - square and cube roots of
the gravitational index is an example \cite{Wessel:1998ve}) in an
automated fashion. While this infrastructure is currently only
supported by the CDK, adoption of it (or some similar framework) would
enable easier interoperability between descriptor implementations and
enhance reproducibility.

\section{Running Notes}
\begin{itemize}
\item \textbf{Done} What are descriptors, why are they needed?
\item Survey of descriptor types, focusing on nature of inputs,
  complexity, interpretabilty 
\item \textbf{Done} Talk about biological descriptors (see paper from Tropsha that
  uses bioactivity data as descriptors)
\item \textbf{Done} Discuss how fingerprints can also be used as descriptors
\item Also mention descriptors for non-small molecule cases
  \begin{itemize}
  \item proteins
  \item polymers \& mixtures
  \item \textbf{Done} inorganics (maybe ask Maciej for pointers)
  \end{itemize}
\item Survey of descriptor implementations (OSS and non-OSS)
  \begin{itemize}
  \item MOE
  \item DRAGON
  \item CDK
  \item RDKit
  \item JChem
  \item PipelinePilot
  \item OEChem/GraphTK
  \item JOELib
  \item CODESSA
  \item Schrodinger stuff (Meastro?)
  \end{itemize}
\item Go into more detail of OSS descriptor implementations
  \begin{itemize}
  \item CDK (describe architecture), mention wrappers around the
    library (CDKDescUI and batch mode)
  \item PaDEL
  \end{itemize}
\item Compare OSS descs to commercial ones - point to the Pfizer/CDD
  paper that compared MOE and CDK
\end{itemize}
\clearpage
\newpage

\bibliography{paper}

\newpage

\ctable[caption={A comparison of Topological Polar Surface Area values
generated using the CDK and ACD Labs software, for 57,600 molecules
taken from Pubchem AID 1996},cap={},label={fig:tpsa},botcap,figure]
{c}
{}
{
  \includegraphics[width=\linewidth]{tpsa-cdk-acd}
}

\ctable[caption={A comparison of experimental versus calculated logP values
generated using the CDK, ACD Labs and ChemAxon software, for 10,000 molecules
taken from the proprietary logPstar dataset [REF?]},cap={},label={fig:logp},botcap,figure]
{c}
{}
{
  \includegraphics[width=\linewidth]{logp-comp-exptl}
}

\end{document}
  
    
     