\documentclass[letterpaper, 12pt]{article}
%%\documentclass[letterpaper, 11pt, twocolumn]{article}
\usepackage{url}
\usepackage{ctable}
\usepackage{overcite}
\usepackage{amsmath,amssymb}      % for \AA etc
\usepackage[onehalfspacing]{setspace}
\usepackage{graphicx}
\usepackage{fullpage, times}
\usepackage{algorithmic}
\usepackage{algorithm}
\usepackage{longtable}

\DeclareGraphicsExtensions{.pdf,.png}
\onehalfspacing

\bibliographystyle{jcics}

\usepackage{anysize}
%\marginsize{1in}{2in}{1in}{1in}
%\marginparwidth 1.75in
%\newcommand{\mnote}[1]{\marginpar{\raggedright{\footnotesize\textbf{#1}}}}

\def\imagetop#1{\vtop{\null\hbox{#1}}}

%%%
%%% Don't change the size of the figures as they are defined for
%%% a single column environment, which is what we will be submitting
%%% as
%%%

\begin{document}
\title{A Survey of Quantitative Descriptions of Molecular Structure}
\author{Rajarshi Guha${}^{\dagger}$ and Egon Willighagen${}^{\ddagger}$\\
${}^{\dagger}$NIH Center for Advancing Translational Science\\9800 Medical
Center Drive  Rockville, MD 20850 \\U.S.A.\\
${}^{\ddagger}$ Department of Bioinformatics - BiGCaT\\Maastricht University\\P.O. Box 616\\NL-6200 MD Maastricht\\The Netherlands
}
\date{}

\maketitle
\begin{abstract}
Bar
\end{abstract}

\section{Introduction}

Computational methods play an important role in many chemical
disciplines ranging from drug discovery to materials science. While
there are a variety of techniques that differ in terms of
computational complexity, time requirements and so on. However the
common requirement underlying all these methods is a formal
description of a the molecular structure. Of course, there are many
ways to ``describe'' a molecule. A common approach is to describe the
connectivity, taking into account the types of atoms and bonds. In
other words, explicit representations of chemical structure such as
SMILES, SD files and so on. While these descriptions are vital to
modern chemical information systems, they do not necessarily allow
computational techniques to be directly applied to them.

Methods that aim to predict chemical and biological properties
generally require a numerical description of chemical structures. Such
numerical forms range from a set of 3D coordinates (which coupled with
appropiate atom types, is sufficient for methods such QM approaches
and docking) to more abstract numerical descriptions derived from 2D
or 3D representations which can be useful is statistical approaches.

It is now possible to evaluate thousands of numerical descriptors of
chemical structure. As will be discussed later, many of these
descriptors are closely related - one can be substituted for
another. The choice of descriptors is a well known problem and given a
large collection of them, approaches to identify a suitable subset  
have been discussed extensively in the literature [REFS]. 

In addition to there being many possible descriptors defined in the
literature, there are also multiple implementations of a give
descriptor. These implementations are available in the form of
libraries (which require one to write a program to use them) or
complete applications (GUI or command line). As a result, not only
must one choose one or more descriptors that are relevant to the
problem at hand, but in addition, one must be concerned about how they
are calculated and whether such a calculation can be reproduced across
different implementations of those descriptors. It is easy to
understand \emph{why} two implementations of the same descriptor can
lead to different results - the primary reason being the chemistry
model that underlies the framework or toolkit used to implement the
descriptor. For example, a descriptor that calculates the number of
aromatic atoms may be implemented using two toolkits with differing
aromaticity models, and hence it is possible that the values generated
by the two implementations will differ. Other sources of differences
include parameters that may be involved in the descriptor calculation
and reference data values (such as atomic radii, electronegativity
values) that are employed during descriptor calculation. While most
implementations will employ the same data sources for standard
concepts (e.g., atomic weights), slight differences in these types of
input data can lead to differences in the final descriptor value. As a
result, in most cases, two implementations of a descriptor do not
usually give the exact same value, though they are usually quite
similar. Explicitly explaining the differences may or may not be
possible (it is usually more difficult in the cases of commercial
implementations for which source code is not available).

\section{A Categorization of Descriptors}
\label{sec:categ-descr}

We have noted that a molecular structure can be characterized using a
numebr of numerical descriptors. In this section we describe a
categorization of descriptors. Admittedly, the grouping is somewhat
arbitrary but serves as a useful guide. In addition, the
categorization also considers the nature of the chemical structure
being considered - some descriptors are only useful when applied to
small molecules, whereas other descriptors are defined specifically
for polymers or protein structures.

\textbf{Fingerprints} A final class of descriptors are
fingerprints. Traditionally, these descriptors are represented in the
form of bit strings. Binary fingerprints can be divided into hashed
fingerprints, where substructures (such as paths of length $n$) are
converted to a string representation and then hashed into a randomly
selected bit positions, or keyed fingerprints, where each bit position
corresponds to a unique substructural feature. The nature of the
features can range from simple functional groups (hydroxyl, carbonyl)
and topological substructures (paths, chains, cycles) to atom
environments \cite{Bremser1978,Bender:2004aa} and pharmacophoric
elements \cite{Renner:2006aa}. Depending on the definition,
fingerprints can work with topology only or may require 3D
conformations. The latter class of fingerprints are usually related to
3D pharmacophores and most fingerprint definitions only require
connectivity information.

\textbf{Bioactivity as a descriptor} While the descriptors mentioned
so far are derived from some form of chemical structure, an
alternative approach is to employ observed biological activities of a
molecule as descriptors of that molecule.  This approach was taken by
Sedykh et al \cite{Sedykh:2011fk} who hypothesized that dose response
data points from high throughput dose-reponse assays could be employed
as biological descriptors. They observed that when such descriptors
were combined with traditional chemical structure descriptors, the
predictive performance of models developed using the combined set was
better than those developed using conventional approaches. A similar
approach can be taken using the PASS methodology
\cite{Poroikov:2007aa}, though this is an indirect method - one must
predict the PASS profile from the chemical structure, which can then
be used in subsequent predictive models [REF?].

\section{Descriptor Implementations}
\label{sec:descr-impl}

A variety of descriptor implementations are available, with most
cheminformatics toolkits providing implementations of a common set of
descriptors and fingerprints. Table \ref{tab:impl} summarizes the
descriptor classes implemented by several toolkits and applications. 

Given the multiple implementations of many descriptors, it is natural
to compare their performance. As noted previously, it is desirable,
but unlikely that implementations of the same descriptor using
different cheminformatcs toolkits will be identical. Gupta et al
employed SMARTS based descriptors from the CDK and MOE to develop
decision tree models to predict human liver microsomal metabolic
stability\cite{Gupta:2010uq}. Their results indicated very similar
performance between the two implementations. 

One can also directly compare descriptor values generated using
different implementations. 

A number of physicochemical descriptors commonly employed in
predictive modeling are themselves models of an experimental
property. For example, log P can be experimentally measured and a
number of algorithms have been developed to predict log P
[REFs]. Given the utility of log P in drug discovery, a number of
implementations are available. Given that this descriptor is a
surrogate for an actual experimental property, it is reasonable to
compare calculated log P values from different implementations to the
experimentall observed values, rather than between themselves. Figure
\ref{fig:logp} compares computed log P values from the CDK, ChemAxon
and ACD Labs for a set of 10,000 molecules taken from the logPstar
dataset [REF].

\section{Running Notes}
\begin{itemize}
\item \textbf{Done} What are descriptors, why are they needed?
\item Survey of descriptor types, focusing on nature of inputs,
  complexity, interpretabilty 
\item \textbf{Done} Talk about biological descriptors (see paper from Tropsha that
  uses bioactivity data as descriptors)
\item Discuss how fingerprints can also be used as descriptors
\item Also mention descriptors for non-small molecule cases
  \begin{itemize}
  \item proteins
  \item polymers
  \item inorganics (maybe ask Maciej for pointers)
  \end{itemize}
\item Survey of descriptor implementations (OSS and non-OSS)
  \begin{itemize}
  \item MOE
  \item DRAGON
  \item CDK
  \item RDKit
  \item JChem
  \item PipelinePilot
  \item OEChem/GraphTK
  \item JOELib
  \item CODESSA
  \item Schrodinger stuff (Meastro?)
  \end{itemize}
\item Go into more detail of OSS descriptor implementations
  \begin{itemize}
  \item CDK (describe architecture), mention wrappers around the
    library (CDKDescUI and batch mode)
  \item PaDEL
  \end{itemize}
\item Compare OSS descs to commercial ones - point to the Pfizer/CDD
  paper that compared MOE and CDK
\end{itemize}
\clearpage
\newpage

\bibliography{paper}

\newpage

\ctable[caption={A comparison of Topological Polar Surface Area values
generated using the CDK and ACD Labs software, for 57,600 molecules
taken from Pubchem AID 1996},cap={},label={fig:tpsa},botcap,figure]
{c}
{}
{
  \includegraphics[width=\linewidth]{tpsa-cdk-acd}
}

\ctable[caption={A comparison of experimental versus calculated logP values
generated using the CDK, ACD Labs and ChemAxon software, for 10,000 molecules
taken from the proprietary logPstar dataset [REF?]},cap={},label={fig:logp},botcap,figure]
{c}
{}
{
  \includegraphics[width=\linewidth]{logp-comp-exptl}
}

\end{document}
  
    
     