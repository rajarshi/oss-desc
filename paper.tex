\documentclass[letterpaper, 12pt]{article}
%%\documentclass[letterpaper, 11pt, twocolumn]{article}
\usepackage{url}
\usepackage{ctable}
\usepackage{overcite}
\usepackage{amsmath,amssymb}      % for \AA etc
\usepackage[onehalfspacing]{setspace}
\usepackage{graphicx}
\usepackage{fullpage, times}
\usepackage{algorithmic}
\usepackage{algorithm}
\usepackage{longtable}

\DeclareGraphicsExtensions{.pdf,.png}
\onehalfspacing

\bibliographystyle{jcics}

\usepackage{anysize}
%\marginsize{1in}{2in}{1in}{1in}
%\marginparwidth 1.75in
%\newcommand{\mnote}[1]{\marginpar{\raggedright{\footnotesize\textbf{#1}}}}

\def\imagetop#1{\vtop{\null\hbox{#1}}}

%%%
%%% Don't change the size of the figures as they are defined for
%%% a single column environment, which is what we will be submitting
%%% as
%%%

\begin{document}
\title{A Survey of Quantitative Descriptions of Molecular Structure}
\author{Rajarshi Guha${}^{\dagger}$ and Egon Willighagen${}^{\ddagger}$\\
${}^{\dagger}$NIH Center for Advancing Translational Science\\9800 Medical
Center Drive  Rockville, MD 20850 \\U.S.A.\\
${}^{\ddagger}$ Department of Bioinformatics - BiGCaT\\Maastricht University\\P.O. Box 616\\NL-6200 MD Maastricht\\The Netherlands
}
\date{}

\maketitle
\begin{abstract}
Bar
\end{abstract}

\section{Introduction}

Computational methods play an important role in many chemical
disciplines ranging from drug discovery to materials science. While
there are a variety of techniques that differ in terms of
computational complexity, time requirements and so on. However the
common requirement underlying all these methods is a formal
description of a the molecular structure. Of course, there are many
ways to ``describe'' a molecule. A common approach is to describe the
connectivity, taking into account the types of atoms and bonds. In
other words, explicit representations of chemical structure such as
SMILES, SD files and so on. While these descriptions are vital to
modern chemical information systems, they do not necessarily allow
computational techniques to be directly applied to them.

Methods that aim to predict chemical and biological properties
generally require a numerical description of chemical structures. Such
numerical forms range from a set of 3D coordinates (which coupled with
appropiate atom types, is sufficient for methods such QM approaches
and docking) to more abstract numerical descriptions derived from 2D
or 3D representations which can be useful is statistical approaches.

It is now possible to evaluate thousands of numerical descriptors of
chemical structure. As will be discussed later, many of these
descriptors are closely related - one can be substituted for
another. The choice of descriptors is a well known problem and given a
large collection of them, approaches to identify a suitable subset
have been discussed extensively in the literature [REFS]. 

In addition to there being many possible descriptors defined in the
literature, there are also multiple implementations of a give
descriptor. These implementations are available in the form of
libraries (which require one to write a program to use them) or
complete applications (GUI or command line). As a result, not only
must one choose one or more descriptors that are relevant to the
problem at hand, but in addition, one must be concerned about how they
are calculated and whether such a calculation can be reproduced across
different implementations of those descriptors. It is easy to
understand \emph{why} two implementations of the same descriptor can
lead to different results - the primary reason being the chemistry
model that underlies the framework or toolkit used to implement the
descriptor. For example, a descriptor that calculates the number of
aromatic atoms may be implemented using two toolkits with differing
aromaticity models, and hence it is possible that the values generated
by the two implementations will differ. Other sources of differences
include parameters that may be involved in the descriptor calculation
and reference data values (such as atomic radii, electronegativity
values) that are employed during descriptor calculation. While most
implementations will employ the same data sources for standard
concepts (e.g., atomic weights), slight differences in these types of
input data can lead to differences in the final descriptor value. As a
result, in most cases, two implementations of a descriptor do not
usually give the exact same value, though they are usually quite
similar. Explicitly explaining the differences may or may not be
possible (it is usually more difficult in the cases of commercial
implementations for which source code is not available).

\section{Running Notes}
\begin{itemize}
\item What are descriptors, why are they needed?
\item Survey of descriptor types, focusing on nature of inputs,
  complexity, interpretabilty 
\item Talk about biological descriptors (see paper from Tropsha that
  uses bioactivity data as descriptors)
\item Discuss how fingerprints can also be used as descriptors
\item Also mention descriptors for non-small molecule cases
  \begin{itemize}
  \item proteins
  \item polymers
  \item inorganics (maybe ask Maciej for pointers)
  \end{itemize}
\item Survey of descriptor implementations (OSS and non-OSS)
  \begin{itemize}
  \item MOE
  \item DRAGON
  \item CDK
  \item RDKit
  \item JChem
  \item PipelinePilot
  \item OEChem/GraphTK
  \item JOELib
  \item CODESSA
  \item Schrodinger stuff (Meastro?)
  \end{itemize}
\item Go into more detail of OSS descriptor implementations
  \begin{itemize}
  \item CDK (describe architecture), mention wrappers around the
    library (CDKDescUI and batch mode)
  \item PaDEL
  \end{itemize}
\item Compare OSS descs to commercial ones - point to the Pfizer/CDD
  paper that compared MOE and CDK
\end{itemize}
\clearpage
\newpage

\bibliography{paper}

\newpage


\end{document}
  
    
     